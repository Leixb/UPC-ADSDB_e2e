%! TEX root = **/000-main.tex
% vim: spell spelllang=en:

% The domain chosen and a description of the original data sources and the
% variables they provide. Define here the analytical question you want to
% answer.

\section{Context}

\subsection{World Health Organization Data}
% https://www.who.int/data/mortality/country-profile

The WHO Mortality Database comprises deaths registered in national vital registration systems, with the underlying reason for death as coded by the relevant national authority. The underlying explanation for death is defined as “the disease or injury which initiated the train of morbid events leading on to death, or the circumstances of the accident or violence which produced the fatal injury” in accordance with the foundations of the International Classification of Diseases.

The database contains number of deaths by country, year, sex, age group and cause of death as far back from 1950. Data are included only for countries reporting data properly coded according to the International Classification of Diseases (ICD).


\subsection{Demographic Data}
% http://data.uis.unesco.org/Index.aspx?DataSetCode=demo_ds#

The database contains data from 1970 to 2019. The data included for countries are Demographic indicators such as:

\begin{itemize}
    \item Fertility rate, total (births per woman)
    \item Life expectancy at birth, total (years)
    \item Mortality rate, infant (per 1,000 live births)
    \item Population aged in groups (<14 to 14), (15 to 24), (25 to 64) and (\geq 65) (thousands)
    \item Population growth (annual \%)
    \item Prevalence of HIV, total (\% of population ages 15-49)
    \item Rural population (\% of total population)
    \item Poverty headcount ratio at \$3.20 a day (PPP) (\% of population)
    \item General government total expenditure (current LCU)
    \item Total debt service (\% of GNI)
    \item Total population (thousands)
    \item Official exchange rate (LCU per US\$, period average)
    \item DEC alternative conversion factor (LCU per US\$)
    \item Gross Domestic Product (GDP) measures
    \item Gross National Income (GNI) measures
    \item Purchasing Power Parity (PPP) measures
\end{itemize}

With these two data sets, we can identify which demographic and socio-economic indicators increase or decrease different deaths causes per country, see which countries share those characteristics. Additionally, we can predict the number of deaths given those features.